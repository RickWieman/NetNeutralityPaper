%!TEX root = paper.tex

In late 2013, Jan Kr\"{a}mer and Lukas Wiewiorra wrote an interesting paper on net neutrality combined with fairness.~\cite{kramer2013fair} This section is based on their observations, statements and definition of fairness. This section may contain quotes and indirect references.

Generally, fairness is defined as \emph{`a judgment of whether an outcome and/or process to reach an outcome are reasonable, acceptable, or just'}.

The net neutrality debate is essentially about fairness. This debate shows in an impressive way to what extent the complaint of Internet activists can grow. For example, slowed down users, for reasons of fairness, want to be compensated more than what prioritized users are willing to pay additionally.

Fairness can be subdivided in three categories: distributive, procedural and interactional fairness. \textbf{Distributive} fairness is evaluated with respect to the outcome of the allocation of a scarce resource: \emph{Is it fair whose data are prioritized?} \textbf{Procedural} fairness is evaluated with respect to the process by which an allocative outcome was achieved: \emph{Is the QoS mechanism fair?} \textbf{Interactional} fairness refers to the treatment that an individual receives: \emph{Is the ISP fair to me by introducing user tiering?} Can be subdivided in \textbf{interpersonal} fairness and \textbf{informational} fairness.

\subsubsection{Distributive (Equality And Equity)}
Distributive fairness has the principles of \emph{equity} and \emph{equality} as its main underlyings.

The \emph{equity} principle suggests that the allocation received should be based on the individual contribution: those users that are willing to pay more, are rightly entitled to receive a better Internet connection.

By contrast, the \emph{equality} principle suggests that everyone should receive the same allocation, independent of the individual contribution. Hence, by the equality principle, all users should receive the same transmission quality, irrespective of their payments for Internet access.

\subsubsection{Procedural (Transparency)}
Procedural fairness relates to whether the procedure or mechanism that determines the outcome is perceived as fair.

The mechanism by which QoS is granted to some users or data streams is more likely to be perceived as fair, if it is transparent to the users how the prioritization of certain data packets is achieved. Consequently, for users that have a strong desire for procedural fairness, it is important that they understand the logic behind a given QoS mechanism, have detailed information about this mechanism and that this information is verifiable.

% bron: http://www.ocf.berkeley.edu/~raylin/tieredinternet.html
With respect to user tiering, procedural fairness is closely related to the level of transparency that the ISP provides to its users with respect to how and why certain data packets are prioritized. User tiering violates the equal treatment of data transmission of different users and is therefore a violation of NN.

\subsubsection{Interactional (Professionalism And Care)}
Like said, interactional fairness can be subdivided up in two concepts: interpersonal and informational fairness. \emph{Inpersonal fairness} relates to the degree to which people feel that processes and outcomes are explained to them and are reasonably justified (= `professionalism'), whereas the perception of \emph{interpersonal fairness} is influenced by politeness, dignity and respect by the decision maker (= `care').

If an event is assumed to be perceived as positive, professionalism and care are assumed to work in the same direction. For example, when the ISP offers a customer to be upgraded to the priority service under user tiering, he is likely to think that the ISP is professional and cares about him.\\
On the other hand, a different result is expected with regard to negative events. For instance, when the ISP communicates to a best-effort customer that it will prioritize the data packets of the premium customers ahead of its own packets, then this customer may still believe that the ISP is professional, but may doubt that it cares about him as a customer.