%!TEX root = paper.tex

% citaties en info van http://papers.ssrn.com/sol3/papers.cfm?abstract_id=2031034.
Def: Generally, fairness is defined as \emph{`a judgment of whether an outcome and/or process to reach an outcome are reasonable, acceptable, or just'}.

The net neutrality debate is essentially about fairness. This debate shows in an impressive way to what extent the complaint of Internet activists can grow. For example, slowed down users, for reasons of fairness, want to be compensated more than what prioritized users are willing to pay additionally.

Fairness can be categorized in:
\begin{itemize}
	\item \textbf{Distributive}\\
		...is evaluated with respect to the outcome of the allocation of a scarce resource.\\
		\emph{Is it fair whose data are prioritized?}

	\item \textbf{Procedural}\\
		...is evaluated with respect to the process by which an allocative outcome was achieved.\\
		\emph{Is the QoS mechanism fair?}

	\item \textbf{Interactional}\\
		...refers to the treatment that an individual receives.\\
		Can be subdivided in \textbf{interpersonal} fairness and \textbf{informational} fairness.\\
		\emph{Is the ISP fair to me by introducing user tiering?}
\end{itemize}

\subsubsection{Distributive (equality and equity)}
Distributive fairness has the principles of \emph{equity} and \emph{equality} as its main underlyings.

The \emph{equity} principle suggests that the allocation received should be based on the individual contribution: those users that are willing to pay more, are rightly entitled to receive a better Internet connection.

By contrast, the \emph{equality} principle suggests that everyone should receive the same allocation, independent of the individual contribution. Hence, by the equality principle, all users should receive the same transmission quality, irrespective of their payments for Internet access.

\subsubsection{Procedural (transparency)}
Procedural fairness relates to whether the procedure or mechanism that determines the outcome is perceived as fair.

Def: User tiering means there are multiple ranks for each user which determine the Internet speed one will get.

% bron: http://www.ocf.berkeley.edu/~raylin/tieredinternet.html
With respect to user tiering, procedural fairness is closely related to the level of transparency that the ISP provides to its users with respect to how and why certain data packets are prioritized. User tiering violates the equal treatment of data transmission of different users and is therefore a violation of NN.

\subsubsection{Interactional (professionalism and care)}
Like said, interactional fairness can be subdivided up in two concepts: interpersonal and informational fairness. \emph{Inpersonal fairness} relates to the degree to which people feel that processes and outcomes are explained to them and are reasonably justified, whereas the perception of \emph{interpersonal fairness} is influenced by politeness, dignity and respect by the decision maker.