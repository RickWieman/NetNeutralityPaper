%!TEX root = paper.tex

% Intro
Before further elaborating on the issue, we need to develop a deeper understanding of the values underlying the net neutrality debate.

% Technological innovation
First of all, \emph{technological innovation} is at stake. Proponents of net neutrality argue that without it, startups will have a hard time `getting of the ground'. Opponents on the other hand argue that net neutrality inhibits innovation. Net neutrality would reduce \acp{ISP} incentive for investing in network infrastructure. Furthermore, net neutrality prevents engineers from creating new (latency sensitive) applications, such as video conferencing.

% Property value
The common argument against net neutrality is based on \emph{property rights}. The argument goes that since \acp{ISP} own their infrastructure, they have the natural right to decide how to use it. \highlight{We'll} elaborate on this argument in section \ref{sec:arguments_property}.

% Economic development
\emph{Economic and social welfare} is used by both proponents and opponents. Both parties argue that economic and social welfare will be maximized, though the norms to achieve this value are different.

% Fairness/transparancy (no monopoly)?
\ldots

% Access to information and knowledge? Democracy?
\ldots

% Freedom of Expression, i.e. the opportunity for users to openly interact with one another?
\ldots