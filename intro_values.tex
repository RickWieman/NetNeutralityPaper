%!TEX root = paper.tex

% Intro
In ethics, a value refers to what a person or group of people consider important in life \cite{friedman2006value}. Values play a crucial role in the policies that we implement. In this section, we provide an overview of some of the values underlying the net neutrality debate. It is worth mentioning that in most cases proponents and opponents pursue the same values.

\begin{itemize}
% Technological innovation
\item First of all, \emph{technological innovation} is at stake. Proponents of net neutrality argue that without it, startups will have a hard time `getting off the ground'. On the one hand because it can gets harder to target a big audience, on the other hand because of the higher entry barriers. Opponents argue that net neutrality inhibits innovation. Net neutrality would reduce \acp{ISP} incentive for investing in network infrastructure. Furthermore, net neutrality prevents engineers from creating new (latency sensitive) applications, such as video conferencing.

% Property value
\item The common argument against net neutrality is based on \emph{property rights}. The argument goes that since \acp{ISP} own their infrastructure, they have the natural right to decide how to use it. We will elaborate on this argument in section~\ref{sec:arguments_property}.

% Wealth (economic development, social welfare) wealth: provider wilt geld verdienen, eindgebruiker niet teveel geld uitgeven. een gezonde economie is de means om dit te bereiken.
\item \emph{Economic and social welfare} is used by both proponents and opponents. Both parties argue that economic and social welfare will be maximized, though the means to achieve this maximum are different.

% Access to information and knowledge? (accessability) access to information/knowledge: dit is zelfs de waarde die "het Internet" onderlegd.
\item In today's information society, \emph{access to information and knowledge} and the related \emph{freedom of expression} are greatly valued. The World Summit on the Information Society Declaration of Principles, for example, makes specific reference to the importance of the right to freedom of expression.

% Fairness/transparancy (no monopoly)? Democratie: internet inrichten op de manier die de meeste mensen willen. voorkomen dat ISPs teveel macht kunnen uitoefenen (checks & balances)
\item The final value we identified is \emph{fairness}. Fairness is a common value in ethical discussions, so we will not elaborate on its application to net neutrality. In the debate, however, neutrality is often mistakenly associated with fairness. To see why this association fails, imagine your neighbour saturating the network with HD movie packets, while you can hardly get any e-mail packets through. Though this is certainly a neutral approach, it is far from fair.
\end{itemize}