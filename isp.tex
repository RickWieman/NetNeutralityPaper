# Internet Service Provider (ISP)

In deze sectie wordt de relatie tussen ISPs en net neutraliteit geschetst. Vervolgens worden de argumenten van ISPs uiteen gezet.

## Relatie m.b.t. net neutraliteit

Internet Service Providers vormen, samen met Internet Content Providers, de belangrijkste actoren in het debat over netneutraliteit. ISP's zijn over het algemeen tegenstanders van netneutraliteit.

In de jonge dagen van het internet waren de mogelijkheden voor QoS niet optimaal. ISPs hebben daarom gekozen om hun klanten een flat-fee model aan te bieden. Op dit moment zijn er meer mogelijkheden om QoS aan te bieden. Dat zou product differentiatie mogelijk maken, maar dit leidt momenteel tot veel weerstand van gebruikers. ISPs zitten hierdoor als het ware opgesloten in hun klassieke model.

## Argumenten

ISP's merken op dat grote internet content providers (ICPs) zoals Google, Yahoo en Microsoft zo veel data genereren dat dit het netwerk verstopt en daardoor de ervaring van andere
gebruikers nadelig beïnvloedt. In reactie daarop hebben ISPs voorgesteld om ICPs een vergoeding te laten betalen die bedoelt is om het netwerk te onderhouden en uit te breiden. [1] ICPs betalen deze vergoeding doorgaans met tegenzin en blijven zich hard maken voor net neutraliteit.

[//]: # (Maar eigenlijk is dat heel gek, want ICPs)
[//]: # (rekenen aan elkaar geen kosten..!)

Ten tweede voorkomt net neutraliteit dat ISP's de consumentensurplus vangen. Consumentensurplus is het bedrag dat consumenten collectief bereid zijn meer te betalen dan de geldende prijs.

Tenslotte voeren ISP's aan dat net neutraliteit mogelijkheden tot (TODO: prijs/product) differentiatie ontneemt. Doordat alle klanten hetzelfde product wordt aangeboden, is het niet mogelijk investeringen af te stemmen op de vraag van de klanten. Dit zou niet bevorderlijk zijn voor verdere ontwikkeling van het Internet.

[//]: # (Maar waarom hebben ze dan zo lang vastgehouden)
[//]: # (aan het 'klassieke' model?)



[1] http://www.theverge.com/2013/1/19/3894182/french-isp-orange-says-google-pays-to-send-traffic