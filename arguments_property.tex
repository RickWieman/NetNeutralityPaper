%!TEX root = paper.tex

% Introduction to section. Seems OK to me..
In this section, we investigate net neutrality from the property perspective. We'll try to answer the following question: is it morally right to forcibly impose net neutrality on ISPs? In order to answer this question, we'll briefly introduce the concept of property rights. In doing so, we'll explain the importance of property rights to society and introduce John Locke's theory of property. Then we'll make an argument against net neutrality using this theory.

% Why do property rights matter? We introduce Locke's arguments for property rights. (TODO: This should be rewritten and enhanced, and we might even elaborate on the importance of property rights by explaning `property rights as the core of libertarian conception of justice'.
Before expanding on property rights, we first examine why property rights matter. Many philosophers have argued in favor of property rights, the most famous of which John Locke. Locke argues that persons own themselves and therefore own their labor. From this argument Locke derives that when one mixes his labor with something in the state of nature, it becomes his property. This is his so called labor theory of property. Furthermore, Locke argues that the right to individual property is a natural right and that it forms the basis of individual freedom and economic security. Protecting ones property forms the primary reason for people to join together and form societies. [4]

% Labor theory applied to net neutrality
How does this labor theory of property apply to the net neutrality debate? ISPs have invested labor (wealth, effort) in building the infrastructure. Using the Lockean theory, the infrastructure is a property of the ISP. Therefore, the ISP has the right to use it in any way it wants. This means that an ISP has the right to require an additional fee for Voice over IP (VoIP) data. Or ask ICPs a fee for a more direct link. Or simply unplug its infrastructure from the Internet. Any enforcement to use the infrastructure in a specific way would violate the ISP's right to property.

% Objection: ISPs have acquired their property in an unfair way
One could object to the property rights because these rights have been the result of government action. ISPs were reportedly protected by governments in the early days. They were granted a monopoly. This is especially the case for AT\&T, where the US government banned any competition for almost hundred years. [1] Though this complicates the matter even more, we argue that it still does not justify taking possession of the (improperly) obtained property by government and to use it for the good of the public. Furthermore, we argue that because most of the Internet is private property, the `last mile' should not be so as well. We believe that if parts of the Internet were made public and parts were held private, values such as freedom of expression would be at risk more than without net neutrality. We think that it is exactly because of the lack of regulation that the internet has flourished and become what we know today.

% Objection: Without neutrality, there would be censorship
This view of property also counters the argument of censorship. According to Wikipeida, `'Censorship is the suppression of speech or other public communication which may be considered ...`' As the Internet is privately owned, the argument of possible censorship fails. [2] Furthermore, prohibition of censorship is already regulated by laws; laws that ISPs are also subject to. These same laws also regulate other issues that net neutrality tries to solve. We argue that when ISPs and users enter an agreement in which the ISP alienates some of its rights, it is just as long as both parties act according to the agreement. `'Net neutrality regulation, on the other hand, is an attempt to impose, by force, the terms of an agreement mutually consenting parties come to about what to do with their property. That is unjust.'' [3]

% Objection: Internet access is of profound value, we need to protect it (TODO: Werk de analogie uit)
The fact that internet access is of profound value does not justify government force against the ISPs to make net neutrality possible. We use an analogy to illustrate this. By analogy, we could just as well force Google to treat all incoming requests equally.

% Afsluiting
Even though the Internet is of immense value, it is and has never been a commons. It seems to be the case that users sometimes overlook this fact. The cables are privately owned. The exchange points are privately owned. The webservers are privately owned. The datacenters are privately owned. Finally, you probably own the computer you use to access the Internet.


[1] http://objectivistanswers.com/questions/2588/are-objectivists-wrong-about-net-neutrality

[2] http://falkvinge.net/2011/08/02/net-neutrality-and-censorship-its-not-about-property-its-about-the-agreement/

[3] http://techliberation.com/2009/12/20/the-deontological-case-against-net-neutrality-regs/

[4] http://www.sparknotes.com/philosophy/johnlocke/section2.rhtml