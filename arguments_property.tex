%!TEX root = paper.tex

% Introduction to section. Seems OK to me..
We investigate net neutrality from the property perspective and we will try to answer the following question: is it morally right to forcibly impose net neutrality on \acp{ISP}? In order to answer this question, we will briefly introduce the concept of property rights. In doing so, we will explain the importance of property rights to society and introduce John Locke's labor theory of property. Then we will make an argument against net neutrality using this theory.

% Why do property rights matter? We introduce Locke's arguments for property rights. (TODO: This should be rewritten and enhanced, and we might even elaborate on the importance of property rights by explaning `property rights as the core of libertarian conception of justice'. It does not really answer the question of why property rights matter..)
\subsubsection{Why Do Property Rights Matter?}
Before expanding on property rights, we first examine why property rights matter. Many philosophers have argued in favor of property rights, the most famous of them is John Locke. Locke argues that persons own themselves and therefore own their labor. From this argument Locke derives that when one mixes his labor with something in the state of nature, it becomes his property. This is his so called \emph{labor theory of property}. Furthermore, Locke argues that the right to individual property is a natural right and that it forms the basis of individual freedom and economic security. Protecting ones property forms the primary reason for people to join together and form societies.

\subsubsection{Labor Theory Of Property Applied To Net Neutrality}
How does this labor theory of property apply to the net neutrality debate? \acp{ISP} have invested labor (wealth, effort) in building the infrastructure. Using the Lockean theory, the infrastructure is a property of the \ac{ISP}. Therefore, the \ac{ISP} has the right to use it in any way it wants. This means that an \ac{ISP} has the right to require an additional fee for \ac{VoIP} data. Or ask ICPs a fee for a more direct link. Or simply unplug its infrastructure from the Internet. Any enforcement to use the infrastructure in a specific way would violate the \acp{ISP} right to property.

% Objection: ISP have acquired their property in an unfair way
One could object to the property rights because these rights have been the result of government granted monopolies. \acp{ISP} were reportedly protected by governments in the early days and that made it possible for them to flourish. In the case of AT\&T, for example, the US government banned any competition for almost one hundred years~\cite{oa2014objectivists}. Though this makes it even more complicated, we argue that it still does not justify taking possession of the (improperly) obtained property by government intervention and to use it for the good of the public.

% Objection: Without neutrality, there would be censorship
Proponents argue that a lack of net neutrality may lead to censorship. If we view the infrastructure as property owned by the \ac{ISP}, we can immediately derive a counterargument. According to Wikipedia, censorship is defined as ``the suppression of speech or other public communication which may be considered ...'' The Internet is private property, thus it cannot be seen as public communication~\cite{falkvinge2014agree}. Furthermore, prohibition of censorship is already regulated by laws in most countries. Accepting net neutrality to guarantee free speech would be redundant.

% Objection: Internet is already non-neutral
%- The Internet already favors application data above latency-sensitive data because of the way its protocols work.

%- Already not a level playing field: big companies are able to distribute their content and thereby achieve a performance advantage. A recent example is the agreement between Netflix and Comcast. This agreement provides Netflix with a better connection. It has been reported that this agreement is not a net neutrality violation. [1] Net neutrality does not address this issue, but the same values are at stake. This provides further evidence that net neutrality is probably not the right means.

%The fact that internet access is of profound value does not justify government force against the \acp{ISP} to make net neutrality possible. By analogy, We use an analogy to illustrate this. By analogy, we could just as well force Google to treat all incoming requests equally. % TODO: Zie "Counterweight to server-side non-neutrality" op Engelse wikipedia van Net Neutrality.

% Afsluiting
Even though the Internet is of immense value, it is and has never been a commons. The cables are privately owned. The exchange points are privately owned. The datacenters are privately owned. Finally, you probably own a computer that you use to access the Internet. It seems that this fact is sometimes overlooked. An \ac{ISP} and a user enter into an agreement in which the \ac{ISP} alienates some of its rights. Forcibly imposing the terms of this agreement would be unjust~\cite{tech2014deon}.