\documentclass{article}

\title{Netneutraliteit}
\author{Martijn Dwars \and Jason Raats \and Tim Rensen \and Rick Wieman}
\date{\today}

\begin{document}

\maketitle

\section{Wie?}
\begin{itemize}
\item Klant/gebruiker
\item Internetaanbieder/provider (ISP)
\item Contentproviders (Youtube + Netflix)
\item Politiek
\item Regulators
\end{itemize}

\section{Morele aspecten}
\begin{itemize}
\item Toegang tot internet is een mensenrecht
\item Mogen ISP's diensten om extra geld vragen om zekerheid te bieden?
\end{itemize}

\section{Feiten}
\begin{itemize}
\item Video diensten gebruiken (veel) meer data dan andere soorten diensten
\item Wet in NL en EU
\end{itemize}

\section{Juridische aspecten}
\begin{itemize}
\item Artikel 7.4A in NL
\item Rechtzaken in Amerika (Verizon en Google)
\end{itemize}

\section{Vraagstelling}
Mogen ISP's diensten discrimineren om zo betere kwaliteit te kunnen garanderen?

\end{document}

% Alleen NL of internationaal
% NL of Engels (Rick)
% Alle mannen mogen stemmen - neutraal
% Verschil tussen privacy en neutraliteit
% Extra aandacht aan morele aspecten
% Wat is neutraliteit
% Economische, politieke aspecten
% Jason en Martijn presenteren