%!TEX root = paper.tex
Welfare can be described in many ways, but in this paper we will only look at the social aspect of it. It is difficult to give clear statements about social welfare, because there are many factors that come into play when we speculate of a system without net neutrality. For example, there are three actors (the \acp{ISP}, \acp{ICP} and the end users) we have to take into account.

The paper of Musaccio~\cite{musacchio2009} tried to put these actors in a statistical model to predict the outcome of two different systems. The first system is an Internet with net neutrality (one-sided pricing) and the second is a system without net neutrality (two-sided pricing). When net neutrality is not an issue, \acp{ISP} can ask both the end users and the \acp{ICP} to pay. This is why they called it the two-sided pricing. Proponents of the net neutrality debate think this is a bad idea, because it would be much harder for \acp{ICP} to innovate and then the end customers would have less social welfare than right now. The customers cannot use the services of some \acp{ICP}, because the costs would be too high. However, there is a situation where all the actors can benefit when there is no net neutrality.

It is true that when net neutrality is not mandatory anymore, \acp{ISP} can choose how much \acp{ICP} must pay for their access to the Internet. Especially in the beginning, \acp{ICP} will face a hard time distributing their services to the end customers. But when all the \acp{ISP} ask too much money, it will not only hold back the innovations of the \acp{ICP}, but also the income of the \acp{ISP} themselves. End users are less likely to use all the services that they were used to, because they have to pay for each service. The services that exist today will be more expensive than right now and the new services can only exist if they have a good financial strategy. This will cause a decrease in data usage and thus less income for the \acp{ISP} in the long run. \acp{ISP} will have to invest in certain \acp{ICP} to differentiate themselves with others to keep the interest of end users. So now the \acp{ICP} are getting (financial) help and the costumers can choose which \ac{ISP} has the best services available. This will lead to more innovation from both the \acp{ISP} and the \acp{ICP}, instead of only the \acp{ICP} in a system with net neutrality. The end users will get the best services with the best network the way they want. The gamers can buy a plan with good latency, the average user can watch YouTube in the quality he wants and the least data-hungry users can send messages without paying too much. This will lead to a greater social welfare than net neutrality can offer.

The only downside to this story is that it is difficult to make predictions. Like we said in the first paragraph, there are many factors that can change the outcome of these changes. There are situations where the one-sided pricing method is preferable, especially when \acp{ISP} really want to make money out of the two-sided pricing model and destroy the faith of the \acp{ICP} and end users. But we think \acp{ISP} know this as well and will support \acp{ICP} in the end to save the growing market and themselves.
