%!TEX root = paper.tex
Welfare kan op verschillende manieren beschreven worden, in deze paper zullen we het tot het sociale aspect bezig houden. Op het gebied van sociale welfare is het lastig om uitspraken te doen, omdat er veel factoren zijn die meespelen in het debat. Zo zijn er bijvoorbeeld de \acp{ISP}, de \acp{ICP} en de eindgebruikers waar rekening mee gehouden moet worden.

In \cite{musacchio2009} zijn veel factoren opgenomen. Er is gekeken hoe een eenzijdig prijsmodel, waarbij \acp{ISP} alleen geld vragen van klanten (netneutraliteit), verschilt met een tweezijdig prijsmodel, waarbij \acp{ISP} zowel van \acp{ICP} als van klanten geld vragen (geen netneutraliteit). Ze hebben met een open blik gekeken naar het probleem en geprobeerd beide situaties zo goed mogelijk na te bootsen m.b.v. modellen. Het is dan ook verrassend om te zien dat het tweezijdig prijsmodel in sommige situaties voor veel partijen (\acp{ISP}, \acp{ICP} en eindgebruikers) preferabel is. Het is namelijk zo dat het na verloop van tijd mogelijk is dat \acp{ISP} \acp{ICP} gaan betalen om innovatie aan de \ac{ICP} kant te bevorderen. Aan deze situatie wordt bijna niet gedacht in het NN-debat, want dan gaat men ervanuit dat alleen de \acp{ICP} extra moeten gaan betalen. Maar dit kan logisch afgeleid worden door de volgende gedachte.

Wanneer \acp{ISP} geld gaan vragen aan \acp{ICP}, dan daalt de innovatie aan de \ac{ICP} kant. Dit is nadelig voor de eindgebruikers, want die krijgen minder kwaliteit diensten aangeboden. Ook voor de \acp{ICP} is dit nadelig, want het is lastiger om een nieuwe dienst op te starten. De \acp{ISP} denken dat het voor hun wel goed uitpakt, maar na verloop van tijd zien ze hun winst terugvallen, want de eindgebruikers zullen minder gebruik gaan maken van veel diensten tegelijkertijd. De eindgebruikers zullen alleen nog de beste diensten gaan gebruiken en dat zorgt voor minder dataverkeer en reclame van \acp{ICP}. Vooral als er veel \acp{ISP} zijn die geld vragen aan \acp{ICP} is dit effect enorm. Het zou voor de \acp{ISP} dus beter zijn als ze bepaalde diensten voor zouden trekken, om de interesse van de eindgebruikers niet te verliezen. De \acp{ISP} zullen investeren in (nieuwe) diensten om zich te differenti{\"e}ren, waardoor er weer ruimte komt voor \acp{ICP} om te groeien. Als deze situatie is bereikt, dan is het voor alle drie de partijen gunstig, want de \acp{ISP} kunnen geld vragen aan twee partijen, de \acp{ICP} worden geholpen door \acp{ISP} en de eindgebruikers krijgen de beste kwaliteit diensten.

Het is alleen heel lastig te voorspellen of deze situatie zich ooit gaat voordoen. Zoals al eerder is aangegeven, er zijn veel factoren die meespelen, waardoor de toekomst moeilijk samengevat kan worden in een analytisch model. Nevertheless, deze theorie biedt een nieuw perspectief, waardbij het niet altijd zo hoeft te zijn dat \acp{ICP} altijd de dupe zullen zijn van een systeem zonder netneutraliteit en dat de drie partijen allemaal een betere sociale welfare bereiken. Zolang de \acp{ISP} niet teveel geld vragen en de hoeveelheid \acp{ISP} niet te veel wordt, is deze situatie niet onmogelijk.