\documentclass{article}

\title{Netneutrality - End users}
\author{Jason}
\date{\today}

\begin{document}
\maketitle

\section{Relatie met Net Neutraliteit}
De groep \emph{end users} is waar het geld vandaan komt. Zowel de ISP's als de ICP's verdienen geld door hun diensten aan deze groep aan te bieden. Een (indirect) gevolg van NN is de lage prijs die ICP's kunnen vragen aan de end users.

\section{Voor NN}
De end users zijn eigenlijk over het algemeen voor NN. De argumenten die hierbij worden gegeven zijn:

\begin{enumerate}
\item Ze willen zo min mogelijk betalen, maar wel alle diensten kunnen gebruiken.
\item Ze zijn bang dat zonder NN de ISP's (of de regulators) te veel macht krijgen. De ISP's zouden kunnen bepalen voor welke diensten meer betaald moet worden. De regulators kunnen makkelijker aangeven welke content niet bekeken mag worden.
\end{enumerate}

Deze meningen worden door de ICP's in de mond gelegd. De meeste gebruikers hebben weinig kennis over dit onderwerp.

\section{Tegen NN}
Op dit moment zijn er weinig tot geen gebruikers die tegen NN zijn. Ze zouden tegen NN zijn als de volgende dingen zouden gelden:

\begin{enumerate}
\item Als zou blijken dat de gemiddelde gebruiker minder zou moeten betalen.
\item Als zou blijken dat de service die de ISP's geeft uiteindelijk beter wordt.
\end{enumerate}

Op dit moment is het moeilijk te zeggen of dit gaat gebeuren zonder NN. Op dit moment is het zo dat er extra betaald moet worden voor bepaalde diensten (bijv. Spotify), maar dat wil nog niet zeggen dat het normale abbonement in prijs zal verlagen als ISP's deze methode mogen doorzetten.

\end{document}


