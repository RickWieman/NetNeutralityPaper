%!TEX root = paper.tex
The net neutrality debate is a complicated one, because many arguments fail to address the right values. This is not surprising, as there is no clear definition of net neutrality. We have tried to come up with the best definition in section~\ref{sec:intro_def}, but we know there is no best answer. We think the first vision of net neutrality (``Maximally useful public information network aspires to treat all content, sites, and platforms equally'') of Tim Wu~\cite{wu2003network} is still the best one.

Some of the proponent's arguments are just. Without net neutrality, some of the values underlying these arguments, such as transparancy and technical innovation, could be at risk. We differ from net neutrality proponents in that we do not think that net neutrality is the right way to secure these values. It is important to strive to the right goal with the right tools. We think proponents of net neutrality want the right goal, but not with the right tool. We have given a couple of examples in sections~\ref{sec:arguments} and~\ref{sec:objections}.

An all-or-nothing approach is not likely to last, because it ignores the delicate aspects of the Internet. Instead, we think that the debate should focus more on a hybrid approach with checks and balances.