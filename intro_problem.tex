%!TEX root = paper.tex
Voordat er ISP's bestonden was het noodzakelijk om een account te hebben bij een universiteit of overheid om gebruik te maken van het vroege internet. Er bestond nog geen browser om informatie op te halen, dus was kennis van Unix vereist. Dit zorgde ervoor dat er maar weinig mensen met het internet werkte. De mensen die het internet commercieel wilden gebruiken waren verplicht de data gratis te verwisselen met de andere gebruikers. Het internet was opgericht met het idee dat universiteiten en overheden snel met elkaar informatie konden wisselen. Het systeem was hierop aangepast en het protocol dat hierbij gebruikt werd zit anders in elkaar dan het protocol dat we tegenwoordig gebruiken.

Het protocol dat tegenwoordig gebruikt wordt heet TCP/IP. Dit protocol werd in rond 1973 ontworpen en was de opvolger van het TCP protocol. Deze ontwikkeling was nodig doordat er in deze tijd meer en meer computers aan elkaar werden gelinkt. Het netwerk moest bestand zijn tegen opstoppingen en moest er ook voor zorgen dat meerdere netwerken met elkaar verbonden konden worden. Het was nu mogelijk om elke computer met elke andere computer te verbinden, waardoor de enorme ontwikkelingen van het internet mogelijk werden.

Tot 1994 was het in Amerika nog niet mogelijk om een eigen provider op te richten. De eerste vier providers die werden aangewezen door de National Science Foundation om publieke access points te bouwen, waren WorldCom, Pacific Bell, Sprint en Ameritech. Deze vier private bedrijven waren gehuisvest in Washingon, San Fransisco, New Jersey en Chicago, waardoor de drukst bevolkte delen in Amerika werden ondersteunt. Het netwerk werd steeds drukker en drukker, totdat het netwerk van de vier verstopt raakten. De bedrijven bouwde zelf meer en snellere access points en gingen gratis samenwerken met kleinere ISPs. Deze constructie hield het niet lang vol, in 1997 sloten de grote ISPs contracten af zodat kleinere ISPs moesten betalen om gebruik te maken van hun netwerk. Omdat de grote ISPs nog geen landelijke dekking hadden, groeide het aantal regionale ISP's elke dag, tot aan 4,000 ISPs in 1997. Dit aantal is in de volgende jaren weer drastisch verkleind, doordat de grote ISPs kleinere ISPs overnamen om zo een landelijke dekking te krijgen. Het werd steeds moeilijker om een ISP op te starten, omdat er veel geld betaald moest worden om snel internet te garanderen. Hierdoor kregen de grote ISPs snel een monopolie.

Dit zorgde voor veel onrust, want de ISPs zouden in theorie kunnen bepalen welke diensten wel en niet werden doorgestuurd. Het concept van een "open internet" werd in het begin van de jaren 2000 bedacht en enkele belangrijke personen pleitte voor dit begrip. Iedereen moest toegang hebben tot alles zonder er extra voor te betalen. Uit dit principe vloeide de term net neutraliteit.