%!TEX root = paper.tex
Before \acp{ISP} existed, it was necessary to be a member of a government or an university to use the Internet. There was no browser or other form of user interface to retrieve data, so knowledge of Unix was a must. Just a handful of computers could access the Internet and few people knew how to work with it. If a person or company wanted to use the Internet for commercial purposes, it had to share all the data free of charge with other Internet users. The Internet was invented to exchange data very quick with others, that you could make money of it was not the most important aspect. Things have changed since then.

More and more people wanted to use the Internet. Technology had to evolve, otherwise it was not possible to meet the growing demand. Several new technologies were invented, but the most important one for this paper is the \acf{IP}. This protocol was invented in 1974 and determines the route of packets that have to be send in a way the path from `A' to `B' is the most effective one. Some packets are sent in different routes, paused or even dropped without the knowlegde of the user. This is an important aspect in the net neutrality debate, which we will explain later.

\subsection{When became net neutrality an issue?}
It was not possible to start an \ac{ISP} before 1994 in the US. The first \acp{ISP} were chosen by the National Science Foundation to build public access points. The Internet became more popular every day, especially when Microsoft included Internet Explorer with its Operating System. The four biggest \acp{ISP} worked together with smaller \acp{ISP} for free to keep up with the demand. This construction did not last long. The four \acp{ISP} wanted to see money, so they made deals with smaller \acp{ISP} that worked at regional scale. This way, the four biggest \acp{ISP} wanted to have national coverage as fast as possible. At its peak, 4,000 \acp{ISP} worked in the US and Canada in 1997. This number declined fast after many small \acp{ISP} were acquitted by the larger \acp{ISP}. Right now there are only a handful of \acp{ISP} and it is very hard to start a new one. You can say the \acp{ISP} have a monopoly.

This lead to turmoil, because in theory the \acp{ISP} could decide which content they wanted to share or block. The idea of an `open' Internet was founded in the beginning of the year 2000. Everybody had to have access to all services without paying more. From this principle the term `Net neutrality' was born.