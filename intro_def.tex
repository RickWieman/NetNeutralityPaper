%!TEX root = paper.tex

The term \emph{network neutrality} was introduced in 2003 by Tim Wu~\cite{wu2003network}, but he does not give any clear definition. However, on his website \highlight{[cite]} he defines network neutrality as a network design principle, where a ``maximally useful public information network aspires to treat all content, sites, and platforms equally''.

Kr\"amer et al.~\cite{kramer2013net} define net neutrality as prohibiting Internet service providers from speeding up, slowing down or blocking Internet traffic based on its source, ownership or destination.

Wikipedia defines net neutrality as the principle that Internet service providers and governments should treat all data on the Internet equally, not discriminating or charging differentially by user, content, site, platform, application, type of attached equipment, and modes of communication.

We define net neutrality as \ldots

The goal of net neutrality already lies in its definition: detaching \acp{ISP} from the information that is sent through their networks.

Several ethical principles could be linked to net neutrality. These principles will also be discussed later on in this paper.

 nondiscrimination principle
 fairness principle
