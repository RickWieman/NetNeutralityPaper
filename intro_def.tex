%!TEX root = paper.tex

The term \emph{network neutrality} was introduced in 2003 by Tim Wu~\cite{wu2003network}, but he does not give any clear definition. However, on his website~\cite{wu2014network} he defines network neutrality as a network design principle, where a ``maximally useful public information network aspires to treat all content, sites, and platforms equally''. He also adds that not every network has to be neutral, but there could be a neutral public network, whose value depends on its neutral nature. The term network neutrality is nowadays shortened to \emph{net neutrality}, but the essence is the same.

Kr\"amer et al.~\cite{kramer2013net} define net neutrality as prohibiting Internet service providers from speeding up, slowing down or blocking Internet traffic based on its source, ownership or destination.

Wikipedia~\cite{wikipedia2014net} defines net neutrality as ``the principle that Internet service providers and governments should treat all data on the Internet equally, not discriminating or charging differentially by user, content, site, platform, application, type of attached equipment, and modes of communication''. It also mentions different definitions, such as limited discrimination with or without \ac{QoS} tiering and first come first served. It states limited discrimination without \ac{QoS} tiering means \ac{QoS} discrimination is allowed, as long as no special fee is charged for higher-quality service. The version with \ac{QoS} tiering actually allows higher fees for \ac{QoS}, under the condition there is no exclusivity in service contracts. According to Crawford~\cite{uhls2007digital}, the first come first served principle means a neutral Internet must forward packets on a first-come, first served basis, without regard for \ac{QoS} considerations.

Google also has a definition of net neutrality. According to Eric Schmidt~\cite{goldman2010why}, discrimination across different types should be allowed, but discrimination between similar types should not be allowed. For example, you could prioritize voice over video, but no videos over other videos.

However, a clear, unambiguous definition does not exist. In this paper, we will therefore use the definition of Tim Wu~\cite{wu2014network}.

The goal of net neutrality already lies in its definition: detaching \acp{ISP} from the information that is sent through their networks.

Several ethical principles could be linked to net neutrality. These principles will also be discussed later on in this paper.

 % nondiscrimination principle
 % fairness principle