%!TEX root = paper.tex

\subsection{The Lockean Theory}
% Lockean theory is niet terecht in dat wanneer je werk mengt het ineens van jou wordt

\subsection{``End-to-end'' principle}
A common argument in favor of net neutrality is the end-to-end principle. [Leg end-to-end principle uit] \cite{iskander2010end}

Though at first glance this argument seems valid, it has one important weakness: the end to end principle has no intrinsic value. Though the Internet was once designed according to this principle, it is not a valid argument for it to stay that way.

% TODO: Waarom is end-to-end ooit bedacht?

% TODO: Maar end-to-end is wel een mean naar bepaalde ends. Welke ends? Zijn die ends wel gerechtvaardigd?

\subsection{Net neutrality is required for democracy}
% Net neutraliteit als vereiste voor een goed werkende democratie (vanwege de informatie vergaring, freedom of speech)
% Zonder net neutraliteit ontstaat er censuur

% De mogelijkheid is er, maar wij geloven dat providers en eindgebruikers dezelfde doelen hebben. Providers willen een zo volledig mogelijke dienst leveren. Ze zullen daardoor geen gebruikers censureren.

\subsection{\acp{ISP} have conflicting interests}
% ISPs hebben teveel tegenstrijdige belangen om een kwalitatief internet waar te maken zonder net neutraliteit. Bijv: waarom zou een ISP WhatsApp niet extreem duur maken, zodat gebruikers weer meer gaan SMSen?

\subsection{\acp{ISP} get too many power}
% ISPs krijgen teveel macht en het internet is daarvoor te belangrijk.

\subsection{Net neutrality as a commons}
% Commons

