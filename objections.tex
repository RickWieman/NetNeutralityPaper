%!TEX root = paper.tex

In this section, we will give several possible objections to our arguments from the previous section.
We will also discuss to what extent these objections hold.

\subsection{The Lockean Theory}
Locke states that you get the ownership over an object through your labor~\cite{tuckness2012locke}.
This can also be applied to the cables of the \acp{ISP}: the \acp{ISP} have put the cables in the ground, so they own the cables.
However, the government also put subsidies into these projects to stimulate the development of the Internet (for example, AT\&T received \$100 million from the FCC for improving their connections~\cite{bode2013att}).
One could therefore ask whether the \acp{ISP} really are the owners of the cables.
According to Locke, they are, because they put labor into it.

Using that theory, Locke states that when you mix your labor with something currently unowned, that object automatically becomes your property.
This was easily refuted by Robert Nozick, giving the famous example of mixing tomato juice with the ocean~\cite{tuckness2012locke}.
Using Lockes theory, this would automatically make you the owner of the ocean, which is of course ridiculous.

This example undermines Lockes theory about ownership, resulting in the question whether the \acp{ISP} really are the owners of the cables, regardless of their labor.
If the \acp{ISP} are no longer the real owners of the cables, our ownership argument against net neutrality falls.

However, we argue that the \acp{ISP} actually are the owners of the cables, not only by the labor they put into it.
The Dutch Ministry of Infrastructure and the Environment (Rijkswaterstaat)~\cite{rws2014kabels} supports our statement, stating that almost all cables are owned by commercial companies.

\subsection{``End-to-End'' Principle}
A common argument in favor of net neutrality is the end-to-end principle, as also given by Iskander et al.~\cite{iskander2010end}.
The end-to-end principle is (according to them) one of the system principles of the Internet, describing that network features should be as close to the end points of the network as possible.
This means that application specific functions should not be inside the network, but (when possible) at the end points.
The main purpose of this is keeping the network as fast as possible by keeping the lowest layers as simple as possible.
Proponents of net neutrality use this principle by stating that the network should accordingly be neutral or `dumb'.

Though at first glance this argument seems valid, it has one important weakness: the end-to-end principle has no intrinsic value. Though the Internet was once designed according to this principle, it is not a valid argument for it to stay that way.
It should not be a restriction to change a system in order to improve it significantly; there has to be room for innovation when possible.

The end-to-end principle is also a rather old one.
Saltzer et al.~\cite{saltzer1984end} already discussed its effectiveness in their paper of 1984.
For example, they argue the end-to-end principle holds for a real time phone conversation over Internet, but that it does not for recording that same conversation.
In other words, the principle is not some kind of holy grail that should be applied whenever possible.

When we add net neutrality to the equation, it becomes clear that the correlation with the end-to-end principle is very low.
Net neutrality does not influence the original goals of the end-to-end principle, as these goals can also be reached without it, as was the case before the net neutrality law passed.
Therefore, we do not believe the end-to-end principle is a valid argument in favor of net neutrality.

\subsection{Net Neutrality Is Required For Democracy}
Another argument for net neutrality could be that it is required for democracy.
Proponents (such as Watkinson~\cite{watkinson2012democracy}) argue that net neutrality ensures freedom of speech and the access to information.
They think that without net neutrality, the Internet would be censored or filtered.

We think the Internet is indeed stimulating freedom of speech and democracy.
However, the Internet is something different than net neutrality.
We do not believe net neutrality is \emph{the} means for ensuring democracy or freedom of speech on the Internet, because there is no reason why net neutrality would guarantee them.
Additionally, the Internet can still be democratic and allow freedom of speech without net neutrality.
The two simply do not exclude each other.

\subsection{\acp{ISP} Get Too Much Power}
Some proponents of net neutrality think the \acp{ISP} will get too much power if we do not regulate net neutrality.
They argue the \acp{ISP} are going to pretend to be some sort of god, because they will be able to decide what gets served and what not. % ref?

Firstly, we do not see why too much power would equal bad behavior.
For example, Google currently has a lot of power, but should we therefore block access to that?
And what about the government?
They also have a lot of power, but still we follow their rules without protesting.

Secondly, net neutrality is in our opinion not \emph{the} way to impose rules.
We also believe that without net neutrality, the system will be more transparant;
Net neutrality would silently enforce \acp{ISP} to behave according to certain rules, and because of that, nobody would really care about the why and how.
Without net neutrality, people probably want to watch the behavior of \acp{ISP} more closely, resulting in more or less compelled transparancy.
Because of this transparancy, the `bad' power of \acp{ISP} (if any) will be very limited, or even negligible.

\subsection{The Internet Is A Commons}
Proponents of net neutrality sometimes argue the Internet is a commons and should therefore be neutral.
According to Wikipedia~\cite{wikipedia2014commons}, a commons ``refers to the cultural and natural resources accessible to all members of a society''.
You could indeed argue that the Internet fits perfectly well in this definition, and this is also mentioned by Barnes~\cite{barnes2003capitalism}.

However, we believe the Internet was never designed as a commons.
As mentioned in section~\ref{sec:intro_problem}, the Internet started small and it was only accessible for a selective group of people.
During the last years, the Internet grew enormously and it became easier accessible for many people.
However, all the parts of the Internet (data centers, hubs, cables, et cetera) are still privately owned.
Barnes~\cite{barnes2003capitalism} acknowledges this (however stating that only \emph{some} parts are privately owned), but he also states the Internet as a whole system is a commons.
We disagree on this, because we think a commons cannot be something that is completely privately owned.