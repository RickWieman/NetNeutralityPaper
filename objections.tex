%!TEX root = paper.tex

\subsection{The Lockean Theory}
Locke states that you get the ownership over an object through your labor~\cite{tuckness2012locke}.
This can also be applied to the cables of the \acp{ISP}: the \acp{ISP} have put the cables in the ground, so they own the cables.
However, the government also put subsidies into these projects to stimulate the development of the Internet (for example, AT\&T received \$100 million from the FCC for improving their connections~\cite{bode2013att}).
One could therefore ask whether the \acp{ISP} really are the owners of the cables.
According to Locke, they are, because they put labor into it.

Using that theory, Locke states that when you mix your labor with something currently unowned, that object automatically becomes your property.
This was easily refuted by Robert Nozick, giving the famous example of mixing tomato juice with the ocean~\cite{tuckness2012locke}.
Using Lockes theory, this would automatically make you the owner of the ocean, which is of course ridiculous.

This example undermines Lockes theory about ownership, resulting in the question whether the \acp{ISP} really are the owners of the cables, regardless of their labor.
If the \acp{ISP} are no longer the real owners of the cables, our ownership argument against net neutrality falls.

However, we argue that the \acp{ISP} actually are the owners of the cables, not only by the labor they put into it.
The Dutch Ministry of Infrastructure and the Environment (Rijkswaterstaat)~\cite{rws2014kabels} supports our statement, stating that almost all cables are owned by commercial companies.

\subsection{``End-to-end'' principle}
A common argument in favor of net neutrality is the end-to-end principle. [Leg end-to-end principle uit] \cite{iskander2010end}

Though at first glance this argument seems valid, it has one important weakness: the end to end principle has no intrinsic value. Though the Internet was once designed according to this principle, it is not a valid argument for it to stay that way.

% TODO: Waarom is end-to-end ooit bedacht?

% TODO: Maar end-to-end is wel een mean naar bepaalde ends. Welke ends? Zijn die ends wel gerechtvaardigd?

\subsection{Net neutrality is required for democracy}
% Net neutraliteit als vereiste voor een goed werkende democratie (vanwege de informatie vergaring, freedom of speech)
% Zonder net neutraliteit ontstaat er censuur

% De mogelijkheid is er, maar wij geloven dat providers en eindgebruikers dezelfde doelen hebben. Providers willen een zo volledig mogelijke dienst leveren. Ze zullen daardoor geen gebruikers censureren.

\subsection{\acp{ISP} have conflicting interests}
% ISPs hebben teveel tegenstrijdige belangen om een kwalitatief internet waar te maken zonder net neutraliteit. Bijv: waarom zou een ISP WhatsApp niet extreem duur maken, zodat gebruikers weer meer gaan SMSen?

\subsection{\acp{ISP} get too many power}
% ISPs krijgen teveel macht en het internet is daarvoor te belangrijk.

\subsection{Net neutrality as a commons}
% Commons

