\documentclass{article}

\title{Utilitarisme}
\author{Jason Raats \and Tim Rensen}

\begin{document}
\maketitle

\section*{Eindgebruiker}
De grootste groep uit dit onderzoek, waardoor de mening erg veel invloed heeft op het eindordeel. 

Stelling: De gemiddelde gebruiker is beter af zonder netneutraliteit.

Argument: Op dit moment betaalt de gemiddelde gebruiker te veel. Als we kijken naar de vier grootste continenten (Noord-Amerika, Latijns-Amerika, Europa en Azi\"e), dan is te zien dat de helft van alle gebruikers die het minst verbruikt verantwoordelijk is voor minder dan 10\% van het totale dataverbruik. Wanneer we alleen naar het mobiele dataverkeer kijken wordt dit percentage nog lager, namelijk 5\%.

Zonder netneutraliteit kunnen ISP's abbonementen aanbieden waardoor klanten sommige diensten niet kunnen ontvangen. Het is dan bijv. niet mogelijk om YouTube te bekijken (of niet in HD). Als de klant deze dienst niet gebruikt, dan hoeft de klant hier ook niet voor te betalen, waardoor de klant goedkoper uit is.

KV = KleinVerbruikers, GV = GrootVerbruikers

\begin{tabular}{ l| c c c c }
& 50\% KV & 50\% KV & 1\% GV & 1\% GV \\
& (fixed) & (mobile) & (fixed) & (mobile) \\
\hline
Latijns Amerika & 7.5\% & 4.0\% & 20.6\% & 26.1\% \\
Noord Amerika & 6.4\% & 1.4\% & 34.2\% & 22.6\%\\
Europa & 9.5\% & 0.3\% & 30.7\% & 46.0\% \\
Azi\"e & 6.7\% & 4.7\% & 25.5\% & 41.1\% \\
\end{tabular} 

\section*{ISP}
Stelling: De ISP's zijn blijer zonder netneutraliteit.

Argument: De ISP's zijn nu verplicht om elke dienst aan te bieden bij elk contract. Als er geen netneutraliteit is, dan kunnen de ISP's abbonementen aanbieden die beter bij de klanten past. Zijn er veel klanten die wel YouTube kijken, maar geen Netflix, dan kunnen ze zo'n abbonement aanbieden, waardoor de klant goedkoper uit is. Dit lijkt nadelig voor de ISP, maar de ISP hoeft door deze constructie minder geld uit te geven aan zijn infrastructuur. Dus de ISP heeft minder inkomsten, maar heeft ook minder uitgaven.

Ook kunnen de ISP's de minder inkomsten tegengaan door contracten af te sluiten met de ICP's. De ICP's die erg veel data verbruiken, zoals YouTube en Netflix, zouden dan tegen een betaling kwaliteitsgarantie af kunnen sluiten. De ISP kan deze dienst voorrang geven, waardoor de gebruiker een goede kwaliteit binnen krijgt.

\end{document}